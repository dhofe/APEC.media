\documentclass[english]{book}
%\usepackage{amsbsy}

\PassOptionsToPackage{unicode=true}{hyperref}
\usepackage{hyperref}
\hypersetup{
	pdfborder={0 0 0},
	breaklinks=true}

\title{APECpedia}

\begin{document}

\maketitle

\tableofcontents

\addcontentsline{toc}{chapter}{Preface}
\chapter*{Preface} \nonumber

Throughout this text vectors are displayed in bold face.


\part{Physical Models}


\chapter{Quantum Mechanics}

\section{Dirac equation}

\[
\left( i \gamma^{\mu}\partial_{\mu} - m \right) \psi = 0
\]

where

$\gamma^{\mu}$ are the gamma matrices.

\chapter{Classical Mechanics}

\section{Newton's Law of Gravitation}

\[
\mathbf{F} = -G \frac{m'm}{r^3} \mathbf{r}
\]

\chapter{Electromagnetism}

SI units are used throughout this chapter.

\section{Force Fields in Terms of Potential Fields}

\begin{equation}
\mathbf{E}=-\nabla\phi - \frac{\partial\mathbf{A}}{\partial t}              \label{eq:Epot}
\end{equation}

\begin{equation}
\mathbf{B}=\nabla\times\mathbf{A}          \label{eq:Bpot}
\end{equation}


\section{Maxwell's Equations}

Coulomb's law:
\begin{equation}
\nabla\cdot\mathbf{E}=\frac{\rho}{\epsilon}    \label{eq:lawCoulomb}
\end{equation}

Gauss law:
\begin{equation}
\nabla\cdot\mathbf{B}=0        \label{eq:lawGauss}
\end{equation}

Faraday's Law of Induction:
\begin{equation}
\nabla\times\mathbf{E}=-\frac{\partial\mathbf{B}}{\partial t}    \label{eq:lawFaraday}
\end{equation}

Ampere's Law with Maxwell's correction:
\begin{equation}
\nabla\times\mathbf{B}=\mu\mathbf{J}+\frac{1}{c^{2}}\frac{\partial\mathbf{E}}{\partial t}        \label{eq:lawAmpere}
\end{equation}

Constitutive Relations (isotropic medium):

\[ \mathbf{D} = \epsilon \mathbf{E} \]

\[ \mathbf{B} = \mu \mathbf{H} \]

\[ \mathbf{J} = \sigma \mathbf{E} \]

Other:

\[ c = \frac{1}{\sqrt{\mu \epsilon}}\]

Where:

$\mathbf{E} =$ Electric field intensity/strength

$\mathbf{H} =$ Magnetic field intensity/strength

$\mathbf{B} =$ Magnetic induction / Magnetic flux density

$\mathbf{D} =$ Electric displacement / Electric flux density

c = velocity of light in the medium

$\sigma =$ Conductivity

Note that if the medium is not isotropic then $\epsilon$ and $\mu$ become tensors. For instance, in the presence of a gravitational field the refractive index of the vacuum changes due to its effect on virtual electron-positron pairs.

When using the Maxwell's equations on problems where we have paths, surfaces and/or volumes changing with time, in such cases the integral form of the equations must be used.


\section{Lorentz Force}

\[
\mathbf{F}=q\left(\mathbf{E}+\mathbf{v}\times\mathbf{B}\right)
\]


\section{Scalar Waves}

Scalar waves result when the electric and magnetic field components are zero, but not so the electric and/or magnetic potentials.

In scalar waves the Lorentz gauge needs not to be zero, but becomes instead an scalar field S~\cite{Vlaen}:

\begin{equation}
S = -\frac{1}{c^{2}}\frac{\partial\phi}{\partial t} - \mathbf{\nabla} \cdot \mathbf{A}      \label{eq:Spot}
\end{equation}

By replacing equations~\ref{eq:Epot}, \ref{eq:Bpot} and \ref{eq:Spot} in Maxwell's equations~\ref{eq:lawCoulomb}-\ref{eq:lawAmpere} we get the two potential wave equations:

\[
\left( \frac{1}{c^{2}}\frac{\partial^{2}\phi}{\partial t^{2}} - \nabla^{2}\phi \right) + \frac{\partial S}{\partial t} = \frac{\rho}{\epsilon}
\]

\[
\left( \frac{1}{c^{2}}\frac{\partial^{2}\mathbf{A}}{\partial t^{2}} - \nabla^{2}\mathbf{A} \right) - \nabla S = \mu \mathbf{J}
\]

If the electric field is zero, then from equation~\ref{eq:Epot} we have~\cite{Dea}:

\begin{equation}
\nabla\phi + \frac{\partial\mathbf{A}}{\partial t} = 0      \label{eq:xi}
\end{equation}

Equation~\ref{eq:xi} can always be satisfied if a scalar field $\chi$ exists such that

\begin{equation}
\mathbf{A} = \nabla \chi      \label{eq:AXi}
\end{equation}

and

\begin{equation}
\phi = - \frac{\partial \chi}{\partial t}        \label{eq:PhiXi}
\end{equation}

If in addition the scalar field S is zero, then by replacing equations~\ref{eq:AXi} and \ref{eq:PhiXi} in \ref{eq:Spot} we get the wave equation for the new scalar field $\chi$:

\begin{equation}
\frac{1}{c^{2}}\frac{\partial^{2}\chi}{\partial t^{2}} - \nabla^{2}\chi = 0
\end{equation}


\chapter{Einstein's Relativity Theory}

\section{Special Relativity}

Composition law for velocities~\cite{dinverno}:

\[
v_{AC} = \frac{v_{AB} + v_{BC}}{1+v_{AB}v_{BC}}
\]


\section{General Relativity}

Einstein's Field Equations:

\[
G_{\mu\nu}=R_{\mu\nu}-\frac{1}{2}Rg_{\mu\nu}=\frac{8\pi G}{c^{4}}T_{\mu\nu}
\]

$R_{\mu\nu}$is the Ricci Tensor

R is the Curvature Scalar

$T_{\mu\nu}$is the Energy-Momentum Tensor


\addcontentsline{toc}{part}{Bibliography}
\begin{thebibliography}{99}
	\bibitem{Vlaen} van Vlaenderen, Koen J. \emph{A generalisation of classical electrodynamics for the prediction of scalar field effects}. 2003 (physics/0305098v1).
	\bibitem{Dea} Dea, Jack. \emph{Fundamental fields and phase information.} Planetary Association for Clean Energy Newsletter, Vol. 4, Number 3.
	\bibitem{dinverno} d'Inverno. \emph{Introducing Einstein's Relativity.} Oxford.
\end{thebibliography}

\end{document}
