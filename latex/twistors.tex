\chapter{Twistor Theory}

References:

\begin{itemize}
	\item "Twistor Theory", R. Penrose, F. Hadrovich. \newline https://web.archive.org/web/20210623180941/https://users.ox.ac.uk/~tweb/00006/index.shtml
\end{itemize}

All the massive fields arise from the interaction of more primordial zero-mass fields with a Higgs field, hence the importance of Twistor Theory.

In relativity, an observer sees a t=const. section of its past null cone. This is a celestial (Riemann) 2-sphere that can be stereographically projected onto a complex plane (spinor space). Möbius transformations of the plane correspond to Lorentz transformations of the celestial sphere.

Thus the relation between spacetime and spinor space is that of a stereographic projection (i.e. inversion) between a sphere and a (complex) plane. Lorentz transformations of the (spacetime) sphere correspond to Möbius transformation of the (spinor) complex plane. Both the Möbius transformations and the stereographic projections are conformal transformations (i.e. angle and orientation preserving), so the outline of a rapidly moving sphere is still perceived as a circle.

A space-time point can be represented as a Riemann sphere in terms of some section of its light cone. This is precisely how space-time points are represented in the projective twistor space.

\begin{verse}
	spacetime -- twistor space \\
	event (point) (past light cone section, i.e. celestial sphere) -- Riemann sphere \\
	light ray (in spacetime) -- point 
\end{verse}

Twistors encode the four-momentum and the six-angular momentum of a massless particle.

Null twistors (zero norm) represent light rays (photons). Twistors with positive (resp. negative) norm represent massless particles ('spinning light rays/photons') with positive (resp. negative) helicity -- i.e. right-handed (resp. left-handed).

Massless field of spin s is a spinor field with 2s indices, primed for s>0 and unprimed for s<0..

The complex conjugate structure is given by a map from twistor space to dual twistor space. This defines a pseudo-norm on twistor space by $Z^\alpha \bar{Z}_\alpha$. Twistors such that this pseudo-norm vanishes are called null twistors and have the property that they are incident with a real $x^a$, and indeed with a whole real null line. For this reason, twistor space can be considered as a complexification of the space of null rays. A remarkable feature of twistor space is that it divides so simply into two halves (of positive and negative pseudo-norm) with the null twistors as the common boundary. Note that to choose twistor space, as opposed to dual twistor space, is to make a strongly chiral choice.

The property of being of positive (or negative) frequency (and hence energy) can be expressed in terms of the geometry of the two halves of twistor space, without any reference to momenta. For such fields, there is an inner product structure, which can be realized in twistor space as a (many-dimensional) compact contour integral, without reference
to space–time coordinates. As such, it is manifestly finite and conformally invariant. Moreover, in the particularly important case of spin 1, it is manifestly gauge invariant.

[Hodges]

