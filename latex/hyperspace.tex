\chapter{Hyperspace}

\section{Introduction}

Some physicists affirm that superimposed to our usual 4D spacetime there is another 4D spacetime which is usually called the dark or negative sector of our universe, the hyperspace or the counterspace. Notable examples of this are:

\begin{itemize}
	\item Jean-Pierre Petit's "Janus cosmological model" (https://januscosmologicalmodel.com/)
	\item H. David Froning Jr. (see his book "Faster than light - Warp drive and quantum vacuum power")
	\item Nick Thomas counterspace (\href{http://nct.goetheanum.org/}{http://nct.goetheanum.org/})
	
\end{itemize}

The set usual 4D spacetime + dark sector is normally called the \emph{dual universe} in the literature. According to many mathematicians, a 5D manifold is equivalent to two 4D manifolds. It has been noted by several physicists that unification of relativity and electromagnetism is most easily achieved by assuming a 5D spacetime.

Thus the normal and dark sectors of the universe would work as the two light beams that interfere (superimpose) on a (4D) photographic plate to make up a (5D) hologram.

Some properties of hyperspace are:

\begin{itemize}
	\item It is the natural home of tachyons, dark matter and dark (or negative) energy.
	\item The only influence that can pass from the positive sector to the negative and vice-versa is gravity. Mass and energy in the dark sector are perceived as negative (or imaginary) mass and energy from the positive sector, and vice-versa. Hence the concept of dark matter and dark energy.
	\item The metric is different in the positive and negative sectors. Both metrics take into account the mass in both sectors, but in different manner because the geometry of the four dimensions is different in each sector.
	\item Time runs in the oposite direction than in the 'positive' 4D sector.
	\item Chirality is reversed.
\end{itemize}

Physical phenomena involving hyperspace include:

\begin{itemize}
	\item Quantum tunneling
	\item Quantum entanglement (see footnote)
	\item Evanescent waves
\end{itemize}

The relation between the 'positive' 4D sector and the hyperspace seem to be that of polar opposites in a stereographic projection\footnote{According to Irving Ezra Segal's article "Theoretical foundations of the chronometric cosmology" (Segal, I. E. (1976) Proc. Natl Acad. Sci. USA 73, 669-673) the total energy of a photon is $E=E_0+E_1$, where $E_0$ is the usual positive relativistic energy, and $E_1$ is the super-relativistic energy, which is the transform of the relativistic energy by conformal inversion. Also according to his article "Covariant chronogeometry and extreme distances: Elementary particles" (Proc. NatL Acad. Sd. USA, Vol. 78, No. 9, pp. 5261-5265, September 1981), $E_0$ is the localized part of the photon's energy, whereas $E_1$ is the delocalized part. This could explain the phenomenon of quantum entanglement.} \\
(\href{https://en.wikipedia.org/wiki/M%C3%B6bius_transformation#Stereographic_projection}{https://en.wikipedia.org/wiki/M\"{o}bius\_transformation\#Stereographic\_projection}).

This has some resemblance also to twistor theory, where there are two dual spaces, the usual 4D spacetime and the 4D spinor space. When you take into account both the positive and negative/dark sectors of the universe then you get two 4D spinor spaces, or in other words, a 8D twistor space.

\section{Conformal Transformations}

In addition to the Poincaré transformations \ref{eq:Poincare} there is a natural extension to an 11-dimensional group by including the scale transformation

\[
x^a \rightarrow \lambda x^a
\]

Remarkably, there is a further transformation (an \emph{inversion}) defined by

\[
x^a \rightarrow \frac{x^a - 2(t^b x_b) t^a}{x^2}
\]

where $t^a$ is a unit timelike 4-vector, with the property that together with the Poincaré and scale transformations it generates the 15-dimensional conformal group. For proper definition the inversion map requires the compactification of Minkowski space with a null cone at infinity.

There is a useful analogue of the conformal transformations with the simpler case of the Möbius transformations on the complex plane: the analogue of the Poincaré transformation is $z \rightarrow e^{i\theta} z + b$, with three real dimensions; scalings enlarge the group to $z \rightarrow az + b$, with four real dimensions, and now the inversion map $z \rightarrow 1/z$, together with these, generates the six-real-dimensional Möbius group. This also requires compactification, but the simpler one of adding a point at infinity to give the Riemann sphere —- or, equivalently, the projective space $CP^1$.

The concept of \emph{angle} is invariant under this group; hence the concept of a Möbius transformation as a conformal (shape-preserving) mapping from the Riemann sphere onto itself.
Likewise, there are conformal invariants in (compactified) Minkowski space. The simplest observation is that the concept of \emph{null separation} is conformally invariant. A conformal transformation maps light cones into light cones. Intuitively, emphasizing conformal symmetry
means that light cones (and so the causal structure of space–time) are regarded as primary, while metric and length scales are secondary.

Zero-rest-mass equations are conformally invariant, e.g. Maxwell's equations without sources (spin 1).
[Hodges]