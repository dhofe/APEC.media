\part{Glossary}

\chapter{Mathematics}

\section{General}

\begin{list}{}{}
	\item \textbf{Theorem:} conclusion deduced from initial suppositions (postulates).
\end{list}

\section{Algebra}

\begin{list}{}{}
	\item \textbf{$\mathbf{Q^* \: or \: Q^{\dagger}}$} Hermitian adjoint of Q (i.e. conjugate transpose).
	
	\item \textbf{Clifford/Geometric algebra:} the algebra of anti-symmetric tensors. Related to the anti-commutative Grassmann algebra and the fermion algebra of the annihilation and creation operators.
		
	\item \textbf{Hermitian matrix:} $Q=Q^*$.
	
	\item \textbf{Ideal:} in ring theory, a branch of abstract algebra, an ideal of a ring is a special subset of its elements. Ideals generalize certain subsets of the integers, such as the even numbers or the multiples of 3. Addition and subtraction of even numbers preserves evenness, and multiplying an even number by an integer results in an even number; these closure and absorption properties are the defining properties of an ideal (a subset of a given set is closed under an operation of the larger set if performing that operation on members of the subset always produces a member of that subset).
	
	\item \textbf{Orthogonal matrix:} a real square matrix whose columns and rows are orthogonal unit vectors, i.e. $Q^T Q=QQ^T=I, Q^T=Q^{-1}$. Its determinant is either +1 or -1.
			
	\item \textbf{Symplectic algebra:} the algebra of symmetric tensors. Related to the commutative Grassmann algebra and the boson algebra of the annihilation and creation operators.
	
	\item \textbf{Unitary matrix:} $Q^{-1}=Q^*$.
\end{list}{}{}


\section{Calculus}

\begin{list}{}{}
	\item \textbf{Spinor:} two-component complex vector (four real numbers).
	\item \textbf{Twistor:} four-component complex vector (eight real numbers).
\end{list}
	

\section{Group Theory}

\begin{list}{}{}
	\item \textbf{Diffeomorphism:} an homeomorphism where both $f$ and $f^-1$ are differentiable.
	
	\item \textbf{Generator:} e.g. Pauli spin matrices are called \emph{generators} of one \emph{representation} of the group SU(2) (namely the representation in terms of 2x2 complex matrices), because any group member can be expressed in terms of them in the form $exp(i\mathbf{\sigma \cdot\theta}/2)$.

	Each representation will have generators in a form suitable for that representation. They could be matrices of larger size, 	for example, or even differential operators. In every representation, however, the generators will have the same 	behaviour when combined with one another, and this behaviour reveals the nature of the group.

	Thus innocent-looking commutation relations may contain	much more information than one might have supposed: 	they are a ‘key’ that, through the use of e.g. $exp(i\mathbf{\sigma \cdot\theta}/2)$, unlocks the complete mathematical behaviour of the group. In Lie group theory these equations describing the generators are called the ‘Clifford algebra’ or ‘Lie algebra’ of the group.
	
	\item \textbf{Homeomorphism:} a ono-to-one mapping $f: X \rightarrow Y$ of topological spaces X onto Y such that $f$ and $f^-1$ are both continuous. X and Y are said to be homeomorphic, i.e. 1-1 and invertable.
	
	\item \textbf{Isomorphism:} two spaces are isomorphic if there exists a homeomorphism $Q:T_p \rightarrow T_q$ such that $Q$ and its inverse $Q^-1$ are differentiable mappings.
	
	\item \textbf{Lie group:} a group wich have a continuous range of members.

	\item \textbf{O(1,3):} the \emph{Lorentz group}. It is the group of all Lorentz transformations of Minkowski (i.e. flat) spacetime. It is six-dimensional. It consists of 3D rotations (3 parameters) and Lorentz "boosts" (3 parameters). Physical theories that have this group symmetry include the kinematical laws of special relativity, electromagnetism, Dirac equation and the Standard Model of particle physics. Lorentz transformations that preserve the direction of time are called \emph{orthochronous} and are denoted by $O^+(1,3)$. Lorentz transformations that preserve (spatial) orientation are called \emph{proper}, as a linear transformation they have determinant +1, and are denoted SO(1,3). The \emph{restricted Lorentz group} is the group of proper orthochronous Lorentz transformations and is denoted $SO^+(1,3)$.
	
	\item $\Re^n:$ the n-dimensional arithmetic (numerical, Cartesian) space. A point in Cartesian n-space is an ordered n-tuple $(x_1,x_2,...,x_n)$. We must distinguish between Cartesian and Euclidean space. They are not the same, as Euclidean space has the additional metric structure.
	
	\item \textbf{Set:} aggregate of elements.
	
	\item \textbf{SL(2,C):} the group of 2 x 2 complex matrices with determinant +1. It is the group of Lorentz transformations of spinors (isomorphic to the group of Möbius transformations). The relationship is a two-to-one mapping since a given Lorentz transformation (in the general sense, including rotations) can be represented by either +M or -M , for M $\in$ SL(2,C). The abstract space associated with the group SL(2,C) has three complex dimensions and therefore six real ones (the matrices have four complex numbers and one complex constraint on the determinant). This matches the 6 dimensions of the manifold associated with the Lorentz group.
	
	\item \textbf{SO(3):} the \emph{special orthogonal rotation group for three dimensions}. It consists of 3x3 matrices of unit determinant and orthogonal (i.e. the transpose is equal to the inverse). It corresponds to all rotations about the origin of three-dimensional Euclidean space under the operation of composition.
	
	\item \textbf{SO(3,1)=SU(2)$\otimes$SU(2):} the group of Lorentz transformations (i.e. 3D rotations and Lorentz "boosts"). The fact that SO(3,1) is a double cover of SU(2) explains why there are two types of  Weyl spinors (right-handed and left-handed, a.k.a. \emph{dotted} and \emph{undotted}).
	
	\item \textbf{SU(2)=Spin(3):} the \emph{special unitary group in two dimensions} = the spin group. It is diffeomorphic to the 3-sphere $\mathbf{S}^3$. It can be identified with the group Sp(1) of unit quaternions. It is the symmetry group of the weak force. It involves the Pauli spin matrices. It is a double cover of O(3) because the quaternions (q) and (-q) represent the same rotation in 3D.
	\item \textbf{SU(2,2)=Spin(2,4):} the Conformal group (see Relativity glossary below).
	\item \textbf{SU(3):} the Lie group of 3x3 unitary matrices with determinant 1 (special unitary group). It is the symmetry group of the strong force, a.k.a. flavour symmetry.
	\item \textbf{SU(n)}: if U(1) represents the space of change of phase of wave functions of only one component, SU(n) represents the space of change of phase for wave functions with n components. Therefore, whereas the elements of U(1) are scalars, the elements of SU(n) are matrices. For that reason, U(1) is commutative (i.e. Abelian) but SU(n) is not (i.e. it's non-Abelian).
	\item \textbf{U(1):} the group of complex numbers of unit modulus -- all numbers of the form $e^{i\alpha} = \cos\alpha + i\sin\alpha$. Furthermore, since $\cos^2 \alpha + \sin^2 \alpha = 1$, the space of these complex numbers is the circle $\mathbf{S}^1$ .
\end{list}{}{}
	

\section{Topology}

\begin{list}{}{}
	\item \textbf{Fiber bundle:} any fiber bundle has the important property that it is \emph{locally} a product space. In addition, a \emph{trivial} fiber bundle is \emph{globally} a product space.
	\item \textbf{Manifold:} an n-dimensional manifold of points is a topological space in which every point has a neighborhood homeomorphic to some open set in arithmetic n-space.
	\item \textbf{Topological Space:} a set X together with a collection of subsets T fulfilling three conditions: a) T contains the empty set and X; b) if several subsets belong to T, then their intersection also belongs to T; and c) the union of any collection of subsets from T also belongs to T.
\end{list}


\section{Geometry}

\subsection{Projective Geometry}

\begin{list}{}{}
	\item \textbf{Möbius transformation:} geometrically, a Möbius transformation can be obtained by first performing stereographic projection from the plane to the unit two-sphere, rotating and moving the sphere to a new location and orientation in space, and then performing stereographic projection (from the new position of the sphere) to the plane. These transformations preserve angles. Since the stereographic projection is equivalent to an inversion, the Möbius transformation of a point in the (complex) plane can be decomposed into a translation + inversion and reflection w.r.t. the real axis + homothety (scaling) and rotation + translation.
\end{list}


\subsection{Differential Geometry}

\begin{list}{}{}
	\item \textbf{Affine connection:} a mathematical object used to compare vectors at different points on a manifold. It is used to calculate the derivative of a geometrical object along a vector field on a manifold. It can be linear or non-linear. Types:
	\begin{list}{}{}
		\item \textbf{Euclidean (a.k.a. metric or metric-compatible):} a connection such that the inner product of any two vectors (which involves the metric) will remain the same when those vectors are parallel transported along any curve. This is equivalent to:
		\begin{itemize}
			\item A connection for which the covariant derivatives of the metric vanish.
			\item A principal connection on the bundle of orthonormal frames.
		\end{itemize}
		
		\item \textbf{Levi-Civita:} the affine connection used in General Relativity. It is the only Euclidean connection with zero torsion (i.e. symmetric in its lower two indices).
		
		\item \textbf{Finslerian:} non-linear connection. It depends not only on the spacetime coordinates but also on the velocity components.
	\end{list}

	\item \textbf{Anholonomy (a.k.a. non-holonomy):} geometric phase. Some quantity S, characteristic of a system, is slaved to certain variables $X_i$, {i = 1, 2,...}
	which are taken around some kind of loop in X-space. If the values $X_i$ return to their original values (that's what is meant by the loop), yet the slaved quantity S fails to return to its original value, the difference between the S values is the geometric phase or anholonomy.
	
	\item \textbf{Cartan calculus:} Exterior calculus of differential forms.

	\item \textbf{Diffeomorphism:} differentiable coordinate transformation that has an inverse.
	
	\item \textbf{Exterior product:} The product "$\wedge$" of vectors. It represents oriented planes, not vectors or directions.
	
	\item \textbf{Exterior differential operator:} a differential operator for differential forms which is equivalent to the gradient of scalars and the divergence and curl of vector calculus.
	
	\item \textbf{Geometry types:}
	\begin{list}{}{}
		\item \textbf{Euclidean:} Geometry on flat (i.e. without curvature) spaces with signature 0.
		
		\item \textbf{Finsler:} phase-spacetime geometry. A ten-dimensional frame bundle is constructed by asigning to each point of spacetime all possible Lorentzian frames, each one defined by three orientation angles and the three components of the relative velocity.
		
		\item \textbf{Lorentzian (a.k.a. pseudo-Riemannian):} Geometry on curved spaces with signature 1.
		
		\item \textbf{Minkowskian:} Geometry on flat spaces with signature 1.
		
		\item \textbf{Riemannian:} Geometry on curved spaces (but without torsion) with signature 0. Some authors consider a Riemannian space any space endowed with a metric.
		
		The only geometries or spaces that most people can visualize are the Euclidean in 1, 2 and 3 dimensions, and the Riemannian in 2 dimensions (the various types of surfaces).
	\end{list}
	
	\item \textbf{Holonomy:} when in classical mechanics a system is evolving under certain constraints, if they are integrable they lead to a reduction in the number of degrees of freedom and are called \emph{holonomic}.

	\item \textbf{Nonholonomic basis:} a basis which is not a coordinate basis. That is, its vectors are not the tangents to the coordinate axes of any coordinate system. Such basis arise for instance in the study of rotating electromagnetic phenomena, where the relativity of simultaneity can not be neglected.
	
	\item \textbf{Outer product:} the product of two tensors involving no contraction.
	
	\item \textbf{Vectors:}
		\begin{list}{}{}
		\item \textbf{Contravariant:} they are also simply called 'vectors'. They are elements of the tangent space. E.g. velocity.
		\item \textbf{Convariant:} ther are also called 'dual vectors' or '1-forms'. They are elements of the cotangent space. E.g. the gradient of a scalar potential.
		\end{list}
\end{list}


\chapter{Physics}

\section{Quantum Mechanics}

\begin{list}{}{}
	\item \textbf{Baryon:} hadron of big mass. Baryons contain an odd number of quarks.
	
	\item \textbf{Bell's theorem:} mathematical inequalities that are fulfilled by classical systems, and are violated by systems displaying quantum non-locality (e.g. a pair of entangled particles).
	
	\item \textbf{Chirality:} It is Lorentz invariant, but it is not a constant of motion for massive particles. For massless particles the helicity is equal to the chirality, and both are Lorentz invariant and are constants of motion.
	
	\item \textbf{Conformal Field Theory:} a particularly symmetric and mathematically well behaved type of  \emph{quantum field theory}. Such theories are often studied in the context of string theory, where they are associated with the surface swept out by a string propagating through spacetime, and in statistical mechanics, where they model systems at a thermodynamic critical point.
	
	\item \textbf{Hadron:} a particle that is affected by the strong force. There are two types: mesons and baryons.
	
	\item \textbf{Helicity:} The projection of the spin onto the direction of linear momentum. In the absence of external forces, it is time-invariant (i.e. it is conserved). However it is not Lorentz invariant.
	
	\item \textbf{Meson:} hadron of middle mass. Mesons contain an equal number of quarks and antiquarks. All of them are unstable, heavier mesons decay to lighter mesons and ultimately to stable electrons, neutrinos and photons.
	
	\item \textbf{Minimal coupling (between a wave function and the electromagnetic field):} the new derivative operation in the wave function Lagrangian depends linearly on the electromagnetic vector potential A but not on its derivatives.
	
	\item \textbf{Operator:} operators map functions onto functions.
	
	\item \textbf{Particle:} quantum excitation of a field. This definition is applicable also to photons, quasiparticles, phonons, excitons, plasmons, etc.
	
	\item \textbf{Quantum Field Theory:} the application of quantum mechanics to physical objects such as the electromagnetic field, which are extended in space and time. In particle physics, quantum field theories form the basis for our understanding of elementary particles, which are modeled as excitations in the fundamental fields. Quantum field theories are also used throughout condensed matter physics to model particle-like objects called quasiparticles.
	
	\item \textbf{Spin:} the \emph{intrinsic angular momentum} generated by a circulating flow of energy in the wave field of the quantum mechanical particle. The spin of a particle always has the same magnitude, but its direction may change. Both the spin magnitude and its components are quantized. The spin and the magnetic moment of a particle can be parallel or antiparallel.
	
	\item \textbf{Zero Point Energy:} the energy of the vacuum at 0ºk. Therefore this energy is not of thermal origin, but consists of random quantum fluctuations.
	
	\item \textbf{Zitterbewegung:} a reciprocating oscillation from positive to negative enegy state in elementary particles that obey relativistic wave equations (e.g. the Dirac equation for electrons and positrons).
\end{list}
	
	
\section{Condensed Matter}

\begin{list}{}{}
	\item \textbf{Caviton:} collapsing Langmuir wave packet in a plasma due to the feedback effect produced by the ponderomotive force.
	
	\item \textbf{Exciton:} Electron-hole pair in a semiconductor.
	
	\item \textbf{Plasmon:} A quantum of plasma oscillation. Just as light (an optical oscillation) consists of photons, the plasma oscillation consists of plasmons. The plasmon can be considered as a quasiparticle since it arises from the quantization of plasma oscillations, just like phonons are quantizations of mechanical vibrations. Thus, plasmons are collective (a discrete number) oscillations of the free electron gas density. For example, at optical frequencies, plasmons can couple with a photon to create another quasiparticle called a \emph{plasmon polariton}. 
	
	\item \textbf{Polariton:} Excitons impose a dipole moment, which combined with the dipole of the electromagnetic field, couples strongly the excitons and the photons. The final result is an \emph{(exciton) polariton}, considered a quasiparticle composed of half-light and half matter, which behaves as a Bose Einstein condensate or superfluid, even at room temperature. This case is referred to as \emph{liquid light}.
\end{list}

\section{Thermodynamics}

\begin{list}{}{}
	\item \textbf{Irreversible transformation:} A transformation of an isolated system where its total entropy increases.
	
	\item \textbf{Isolated system:} A system that is contained in an enclosure through which no heat can be transferred, no work can be done, and no matter nor radiation can be exchanged. Note that the Second Priciple of Thermodynamics applies only to such systems.
	
	\item \textbf{Reversible transformation:} A transformation of an isolated system where its total entropy remains constant.
\end{list}


\section{Electromagnetism}

\begin{list}{}{}
	\item \textbf{Current:} The flow through a given surface or boundary of electric charge per unit of time. Its SI unit is the \emph{ampere}.
	
	\item \textbf{Current density:} The flow of electric charge per unit of surface and per unit of time.
	
	\item \textbf{Diamagnetism}	Diamagnetic substances have low magnetic permeability. Less than unity. They act as divergent magnetic lenses. They avoid the magnetic field line, as though they're being repelled by the applied magnetic field. Some metals are diamagnetic like copper, zinc, silver, gold, antimony, bismuth and mercury. Dielectrics employed in friction machines to
	store electrostatic charge -- like glass, sulphur, rubber -- are also diamagnetic. The hydrogen atom or free radical is paramagnetic. But hydrogen gas is a diamagnetic substance because, normally, the magnetization of one atom cancels out that of the other. An air flame is diamagnetic and is repelled by
	either of the poles of a strong magnet.

	\item \textbf{Dielectric:} An electrically insulating material that can be polarized by an applied electric field.
	
	\item \textbf{Displacement current:} The quantity $\partial D / \partial t$ appearing in Maxwell's equations that is defined in terms of the rate of change of D, the electric displacement field. Displacement current density has the same units as electric current density, and it is a source of the magnetic field just as actual current is. However it is not an electric current of moving charges, but a time-varying electric field. In physical materials (as opposed to vacuum), there is also a contribution from the slight motion of charges bound in atoms, called dielectric polarization. 
	
	\item \textbf{Electric charge:} The capacity of some fundamental particles to produce an electric field around them. Its SI unit is the \emph{coulomb}. Electric charge can be positive or negative.
	\begin{list}{}{}
		\item \emph{Free charge:} Charge not bound to any atomic nucleus.
		
		\item \emph{Bound charge:} Charge bound to an atomic nucleus.
	\end{list}
	
	\item \textbf{Electric charge density:} The amount of electric charge per unit volume.
	
	\item \textbf{Electric field:} The force field that surrounds electrically-charged particles and exerts force on all other charged particles in the field, either attracting or repelling them.
	
	\item \textbf{Ferromagnetism:} It is a special instance of paramagnetism for high permeability substances -- one that involves a “cooperative	alignment” of molecular magnetic domains. There are only 3 elemental ferromagnetic substances: iron, nickel, and cobalt. And they all cluster together between atomic numbers 26 to 28. Ferromagnetic substances are magnetized by the geomagnetic field -- by magnetic induction. The pole of the compass	needle that points North is actually a South pole.
	
	\item \textbf{Gauge theory:} A physical theory that has gauge symmetry, this meaning that the theory has a parameter called the \emph{gauge}, whose value does not affect the value of the measurable (i.e. observable) variables.
	
	\item \textbf{Magnetic field:} The force field corresponding to the magnetic influence between moving electric charges, electric currents, and magnetic materials.
	
	\item \textbf{Magnetic vector potential:} A vector potential whose curl is equal to the magnetic field.
	
	\item \textbf{Magnetism:} the physical phenomenon consisting in the presence of a magnetic field.
	
	\item \textbf{Magnetization:} The vector field that expresses the density of permanent or induced magnetic dipole moments in a magnetic material. The origin of the magnetic moments responsible for magnetization can be either microscopic electric currents resulting from the motion of electrons in atoms, or the spin of the electrons or the nuclei. Net magnetization results from the response of a material to an external magnetic field. Paramagnetic materials have a weak induced magnetization in a magnetic field, which disappears when the magnetic field is removed. Ferromagnetic and ferrimagnetic materials have strong magnetization in a magnetic field, and can be magnetized to have magnetization in the absence of an external field, becoming a permanent magnet. Magnetization also describes how a material responds to an applied magnetic field as well as the way the material changes the magnetic field, and can be used to calculate the forces that result from those interactions. It can be compared to (di-)electric polarization, which is the response of a material to an electric field. Physicists and engineers usually define magnetization as the quantity of magnetic moment per unit volume.
	
	\item \textbf{Paramagnetism}: Paramagnetic substances tend to orient their long axis parallel to the magnetic force vector and are attracted to one of the poles of the field either in parallel or in an anti-parallel	orientation, the parallel orientation being the most frequent. This is also called the lower energy state. Their permeability to magnetic fields is slightly greater than unity so they act like a magnetic lens that makes the lines of force converge. Aluminum, platinum, manganese, and chromium are examples of paramagnetic substances. Iron is paramagnetic only when heated above 786ºC.
	
	\item \textbf{Polarization:}
		\begin{list}{}{}
			\item \emph{Wave polarization}: The direction in which a wave's displacement occurs.
			
			\item \emph{Dielectric polarization}: Charge separation in insulating materials under the influence of an electric field.
		\end{list}
	
	\item \textbf{Potential}
	\begin{list}{}{}
		\item \textbf{Debye potential:} 
		
		\item \textbf{Hertz vector potentials:} An alternative to the usual electromagnetic scalar and vector potentials, the Hertz vector potentials can be used in some cases to simplify the calculation of the electric amd magnetic fields, specially when dealing with antennas and waveguides.
		
		\item \textbf{Scalar electric potential:} Electric potential energy that a unit charge test particle has when it is under the influence of the electric field produced by another charge.
		
		\item \textbf{Scalar potential:} A scalar field whose gradient is a given vector field.
		
		\item \textbf{Vector potential:} A vector field whose curl is a given vector field.
		
		\item \textbf{Whittaker potentials:} Two scalar functions F and G of space and time, in terms of whose second partial derivatives the electric displacement D and the magnetic field intensity H can be expressed~\cite{Whittaker}. Analytical expressions are known for F and G in terms of the trajectory (position as function of time) of the electric charges producing the fields. It can be shown that both F and G obey the homogeneous wave equation in space and time with propagation velocity equal to the speed of light.
	\end{list}
	
	
	\item \textbf{Voltage:} A.k.a. electric potential difference, it is the difference in electric potential (i.e. energy of a unit charge test particle) between two points in space under the influence of the electric field of an/other charge/s. In a static electric field it is defined as the work needed per unit of charge to move a test charge between the two points. Its SI unit is the \emph{volt}.
\end{list}


\section{Relativity}

\begin{list}{}{}
	\item \textbf{Anti-de Sitter space:} the cases of spacetime of constant curvature are de Sitter space (positive), Minkowski space (zero), and anti-de Sitter space (negative). As such, they are exact solutions of Einstein's field equations for an empty universe with a positive, zero, or negative cosmological constant, respectively. It is best known for its role in the AdS/CFT correspondence, which suggests that it is possible to describe a force in quantum mechanics (like electromagnetism, the weak force or the strong force) in a certain number of dimensions (for example four) with a string theory where the strings exist in an anti-de Sitter space, with one additional (non-compact) dimension.
	
	\item \textbf{Conformal group:} the 15-parameter symmetry group consisting of the Poincaré group (10 parameters, see below), dilation (or scale) transformation (1 parameter), and the special conformal (a.k.a. acceleration) transformation (4 parameters).
	
	\item \textbf{Equivalence principle:} It can be summarized as saying that the gravitational mass equals the inertial mass.
	
	\item \textbf{Minkowski (a.k.a. flat) spacetime:} a spacetime where the Riemann curvature tensor is zero. In flat spacetime each observer can relate his reference frames (determined by his set of orthogonal basis vectors) to another observer's reference frames via a simple Lorentz transformation, provided that no forces are acting on the observers (i.e. observers are moving inertially). In the case when observers are not moving inertially, relating the reference frames of the observers can be greatly facilitated by employing the mathematical machinery of torsion and the anholonomic object.
	
	\item \textbf{Poincaré group:} the 10-parameter symmetry group consisting of spacetime translations (4 parameters) and the proper homogeneus Lorentz transformations (6 parameters). The latter consists of 3D rotations (3 parameters) and boosts (3 parameters).
	
	\item \textbf{Signature:} Number of minus signs in the line element. Both in Special and General Relativity the signature is 1.
	
	\item \textbf{Special conformal transformations:} the 4-parameter symmetry group consisting of an inversion plus a translation plus another inversion.
\end{list}
