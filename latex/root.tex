\documentclass[english]{book}
%\usepackage{amsbsy}

\usepackage[utf8]{inputenc}
\usepackage[T1]{fontenc}
\usepackage{hyperref}
\hypersetup{
	colorlinks=true,
	breaklinks=true}

\title{APECpedia}

\begin{document}

\maketitle

\tableofcontents

\addcontentsline{toc}{chapter}{Preface}
\chapter*{Preface} \nonumber

Throughout this text vectors are displayed in bold face.


\part{Physical Models}


\chapter{Quantum Mechanics}

\section{Dirac equation}

\[
\left( i \gamma^{\mu}\partial_{\mu} - m \right) \psi = 0
\]

where

$\gamma^{\mu}$ are the gamma matrices.

\chapter{Classical Mechanics}

\section{Newton's Law of Gravitation}

\[
\mathbf{F} = -G \frac{m'm}{r^3} \mathbf{r}
\]

\chapter{Electromagnetism}

\section{Preliminary remarks}

Coulomb's law is only valid for electrostatics (i.e. non-moving charges). Biot-Savart law is only valid for magnetostatics (i.e. non-moving wires with constant current). For the generalization of both laws for the non-static case see~\cite{Griffiths} page 449, although it is usually easier to calculate the retarded scalar and vector potentials and then to derive them to obtain E and B. For a moving point charge the potentials become the Liénard-Wiechert potentials, see ibid page 454.

In electrostatics and magnetostatics Newton's third law holds, but in electrodynamics it does not, because the electromagnetic fields carry away part of the momentum. Even perfectly static fields can harbor momentum and angular momentum, as long as E × B is nonzero, and it is only when these field contributions are included that the conservation laws are sustained.

SI units are used throughout this chapter.


\section{Maxwell's Equations}

Gauss law:
\begin{equation}
\nabla\cdot\mathbf{E}=\frac{\rho}{\epsilon}    \label{eq:lawCoulomb}
\end{equation}

Gauss law for magnetism:
\begin{equation}
\nabla\cdot\mathbf{B}=0        \label{eq:lawGauss}
\end{equation}

Faraday's Law of Induction:
\begin{equation}
\nabla\times\mathbf{E}=-\frac{\partial\mathbf{B}}{\partial t}    \label{eq:lawFaraday}
\end{equation}

Ampere's Law with Maxwell's correction:
\begin{equation}
\nabla\times\mathbf{B}=\mu\mathbf{J}+\frac{1}{c^{2}}\frac{\partial\mathbf{E}}{\partial t}        \label{eq:lawAmpere}
\end{equation}

Constitutive Relations (isotropic medium):

\[ \mathbf{D} = \epsilon \mathbf{E} \]

\[ \mathbf{B} = \mu \mathbf{H} \]

\[ \mathbf{J} = \sigma \mathbf{E} \]

Other:

\[ c = \frac{1}{\sqrt{\mu \epsilon}}\]

Where:

$\mathbf{E} =$ Electric field intensity/strength

$\mathbf{H} =$ Magnetic field intensity/strength

$\mathbf{B} =$ Magnetic induction / Magnetic flux density

$\mathbf{D} =$ Electric displacement / Electric flux density

c = velocity of light in the medium

$\sigma =$ Conductivity

Note that if the medium is not isotropic then $\epsilon$ and $\mu$ become tensors. For instance, in the presence of a gravitational field the refractive index of the vacuum changes due to its effect on virtual electron-positron pairs.

When using the Maxwell's equations on problems where we have paths, surfaces and/or volumes changing with time, in such cases the integral form of the equations must be used.


\section{Lorentz Force}

\[
\mathbf{F}=q\left(\mathbf{E}+\mathbf{v}\times\mathbf{B}\right)
\]


\section{Force Fields in Terms of Potential Fields}

\begin{equation}
\mathbf{E}=-\nabla\phi - \frac{\partial\mathbf{A}}{\partial t}              \label{eq:Epot}
\end{equation}

\begin{equation}
\mathbf{B}=\nabla\times\mathbf{A}          \label{eq:Bpot}
\end{equation}

Some physicists argue that there is a gauge freedom in choosing the potentials, so that often the \emph{Lorenz gauge} is used:

\[
\frac{1}{c^2} \frac{\partial \phi}{\partial t} + \nabla \cdot \mathbf{A} = 0
\]

\section{Scalar Waves}

Scalar waves result when the electric and magnetic field components are zero, but not so the electric and/or magnetic potentials.

In scalar waves the Lorenz gauge needs not to be zero, but becomes instead an scalar field S~\cite{Vlaen}:

\begin{equation}
S = -\frac{1}{c^{2}}\frac{\partial\phi}{\partial t} - \mathbf{\nabla} \cdot \mathbf{A}      \label{eq:Spot}
\end{equation}

By replacing equations~\ref{eq:Epot}, \ref{eq:Bpot} and \ref{eq:Spot} in Maxwell's equations~\ref{eq:lawCoulomb}-\ref{eq:lawAmpere} we get the two potential wave equations:

\[
\left( \frac{1}{c^{2}}\frac{\partial^{2}\phi}{\partial t^{2}} - \nabla^{2}\phi \right) + \frac{\partial S}{\partial t} = \frac{\rho}{\epsilon}
\]

\[
\left( \frac{1}{c^{2}}\frac{\partial^{2}\mathbf{A}}{\partial t^{2}} - \nabla^{2}\mathbf{A} \right) - \nabla S = \mu \mathbf{J}
\]

If the electric field is zero, then from equation~\ref{eq:Epot} we have~\cite{Dea}:

\begin{equation}
\nabla\phi + \frac{\partial\mathbf{A}}{\partial t} = 0      \label{eq:xi}
\end{equation}

Equation~\ref{eq:xi} can always be satisfied if a scalar field $\chi$ exists such that

\begin{equation}
\mathbf{A} = \nabla \chi      \label{eq:AXi}
\end{equation}

and

\begin{equation}
\phi = - \frac{\partial \chi}{\partial t}        \label{eq:PhiXi}
\end{equation}

If in addition the scalar field S is zero, then by replacing equations~\ref{eq:AXi} and \ref{eq:PhiXi} in \ref{eq:Spot} we get the wave equation for the new scalar field $\chi$:

\begin{equation}
\frac{1}{c^{2}}\frac{\partial^{2}\chi}{\partial t^{2}} - \nabla^{2}\chi = 0
\end{equation}


\chapter{Einstein's Relativity Theory}

\section{Special Relativity}

Composition law for velocities~\cite{dInverno}:

\[
v_{AC} = \frac{v_{AB} + v_{BC}}{1+v_{AB}v_{BC}}
\]


\section{General Relativity}

Einstein's Field Equations:

\[
G_{\mu\nu}=R_{\mu\nu}-\frac{1}{2}Rg_{\mu\nu}=\frac{8\pi G}{c^{4}}T_{\mu\nu}
\]

$R_{\mu\nu}$is the Ricci Tensor

R is the Curvature Scalar

$T_{\mu\nu}$is the Energy-Momentum Tensor


\chapter{Hyperspace}

\section{Introduction}

Some physicists affirm that superimposed to our usual 4D spacetime there is another 4D spacetime which is usually called the dark or negative sector of our universe, the hyperspace or the counterspace. Notable examples of this are:

\begin{itemize}
	\item Jean-Pierre Petit's "Janus cosmological model" (https://januscosmologicalmodel.com/)
	\item H. David Froning Jr. (see his book "Faster than light - Warp drive and quantum vacuum power")
	\item Nick Thomas counterspace (\href{http://nct.goetheanum.org/}{http://nct.goetheanum.org/})
	
\end{itemize}

The set usual 4D spacetime + dark sector is normally called the \emph{dual universe} in the literature. According to many mathematicians, a 5D manifold is equivalent to two 4D manifolds. It has been noted by several physicists that unification of relativity and electromagnetism is most easily achieved by assuming a 5D spacetime.

Thus the normal and dark sectors of the universe would work as the two light beams that interfere (superimpose) on a (4D) photographic plate to make up a (5D) hologram.

Some properties of hyperspace are:

\begin{itemize}
	\item It is the natural home of tachyons, dark matter and dark (or negative) energy.
	\item The only influence that can pass from the positive sector to the negative and vice-versa is gravity. Mass and energy in the dark sector are perceived as negative (or imaginary) mass and energy from the positive sector, and vice-versa. Hence the concept of dark matter and dark energy.
	\item The metric is different in the positive and negative sectors. Both metrics take into account the mass in both sectors, but in different manner because the geometry of the four dimensions is different in each sector.
	\item Time runs in the oposite direction than in the 'positive' 4D sector.
	\item Chirality is reversed.
\end{itemize}

Physical phenomena involving hyperspace include:

\begin{itemize}
	\item Quantum tunneling
	\item Quantum entanglement (see footnote)
	\item Evanescent waves
\end{itemize}

The relation between the 'positive' 4D sector and the hyperspace seem to be that of polar opposites in a stereographic projection\footnote{According to Irving Ezra Segal's article "Theoretical foundations of the chronometric cosmology" (Segal, I. E. (1976) Proc. Natl Acad. Sci. USA 73, 669-673) the total energy of a photon is $E=E_0+E_1$, where $E_0$ is the usual positive relativistic energy, and $E_1$ is the super-relativistic energy, which is the transform of the relativistic energy by conformal inversion. Also according to his article "Covariant chronogeometry and extreme distances: Elementary particles" (Proc. NatL Acad. Sd. USA, Vol. 78, No. 9, pp. 5261-5265, September 1981), $E_0$ is the localized part of the photon's energy, whereas $E_1$ is the delocalized part. This could explain the phenomenon of quantum entanglement.} \\
(\href{https://en.wikipedia.org/wiki/M%C3%B6bius_transformation#Stereographic_projection}{https://en.wikipedia.org/wiki/M\"{o}bius\_transformation\#Stereographic\_projection}).

This has some resemblance also to twistor theory, where there are two dual spaces, the usual 4D spacetime and the 4D spinor space. When you take into account both the positive and negative/dark sectors of the universe then you get two 4D spinor spaces, or in other words, a 8D twistor space.

\section{Conformal Transformations}

In addition to the Poincaré transformations \ref{eq:Poincare} there is a natural extension to an 11-dimensional group by including the scale transformation

\[
x^a \rightarrow \lambda x^a
\]

Remarkably, there is a further transformation (an \emph{inversion}) defined by

\[
x^a \rightarrow \frac{x^a - 2(t^b x_b) t^a}{x^2}
\]

where $t^a$ is a unit timelike 4-vector, with the property that together with the Poincaré and scale transformations it generates the 15-dimensional conformal group. For proper definition the inversion map requires the compactification of Minkowski space with a null cone at infinity.

There is a useful analogue of the conformal transformations with the simpler case of the Möbius transformations on the complex plane: the analogue of the Poincaré transformation is $z \rightarrow e^{i\theta} z + b$, with three real dimensions; scalings enlarge the group to $z \rightarrow az + b$, with four real dimensions, and now the inversion map $z \rightarrow 1/z$, together with these, generates the six-real-dimensional Möbius group. This also requires compactification, but the simpler one of adding a point at infinity to give the Riemann sphere —- or, equivalently, the projective space $CP^1$.

The concept of \emph{angle} is invariant under this group; hence the concept of a Möbius transformation as a conformal (shape-preserving) mapping from the Riemann sphere onto itself.
Likewise, there are conformal invariants in (compactified) Minkowski space. The simplest observation is that the concept of \emph{null separation} is conformally invariant. A conformal transformation maps light cones into light cones. Intuitively, emphasizing conformal symmetry
means that light cones (and so the causal structure of space–time) are regarded as primary, while metric and length scales are secondary.

Zero-rest-mass equations are conformally invariant, e.g. Maxwell's equations without sources (spin 1).
[Hodges]


\chapter{Glossary}

\section{Electromagnetism}

\begin{list}{}{}
	\item \textbf{Current:} The flow through a given surface or boundary of electric charge per unit of time. Its SI unit is the \emph{ampere}.
	
	\item \textbf{Current density:} The flow of electric charge per unit of surface and per unit of time.

	\item \textbf{Dielectric:} An electrically insulating material that can be polarized by an applied electric field.
	
	\item \textbf{Displacement current:} The quantity $\partial D / \partial t$ appearing in Maxwell's equations that is defined in terms of the rate of change of D, the electric displacement field. Displacement current density has the same units as electric current density, and it is a source of the magnetic field just as actual current is. However it is not an electric current of moving charges, but a time-varying electric field. In physical materials (as opposed to vacuum), there is also a contribution from the slight motion of charges bound in atoms, called dielectric polarization. 
	
	\item \textbf{Electric charge:} The capacity of some fundamental particles to produce an electric field around them. Its SI unit is the \emph{coulomb}. Electric charge can be positive or negative.
	\begin{list}{}{}
		\item \emph{Free charge} Charge not bound to any atomic nucleus.
		
		\item \emph{Bound charge} Charge bound to an atomic nucleus.
	\end{list}
	
	\item \textbf{Electric charge density:} The amount of electric charge per unit volume.
	
	\item \textbf{Electric field:} The force field that surrounds electrically-charged particles and exerts force on all other charged particles in the field, either attracting or repelling them.
	
	\item \textbf{Gauge theory:} A physical theory that has gauge symmetry, this meaning that the theory has a parameter called the \emph{gauge}, whose value does not affect the value of the measurable (i.e. observable) variables.
	
	\item \textbf{Magnetic field:} The force field corresponding to the magnetic influence between moving electric charges, electric currents, and magnetic materials.
	
	\item \textbf{Magnetic vector potential:} A vector potential whose curl is equal to the magnetic field.
	
	\item \textbf{Magnetism:} the physical phenomenon consisting in the presence of a magnetic field.
	
	\item \textbf{Magnetization:} The vector field that expresses the density of permanent or induced magnetic dipole moments in a magnetic material. The origin of the magnetic moments responsible for magnetization can be either microscopic electric currents resulting from the motion of electrons in atoms, or the spin of the electrons or the nuclei. Net magnetization results from the response of a material to an external magnetic field. Paramagnetic materials have a weak induced magnetization in a magnetic field, which disappears when the magnetic field is removed. Ferromagnetic and ferrimagnetic materials have strong magnetization in a magnetic field, and can be magnetized to have magnetization in the absence of an external field, becoming a permanent magnet. Magnetization also describes how a material responds to an applied magnetic field as well as the way the material changes the magnetic field, and can be used to calculate the forces that result from those interactions. It can be compared to (di-)electric polarization, which is the response of a material to an electric field. Physicists and engineers usually define magnetization as the quantity of magnetic moment per unit volume.
	
	\item \textbf{Polarization:}
		\begin{list}{}{}
			\item \emph{Wave polarization}: The direction in which a wave's displacement occurs.
			
			\item \emph{Dielectric polarization}: Charge separation in insulating materials under the influence of an electric field.
		\end{list}
	
	\item \textbf{Potential}
	\begin{list}{}{}
		\item \textbf{Debye potential:} 
		
		\item \textbf{Hertz vector potentials:} An alternative to the usual electromagnetic scalar and vector potentials, the Hertz vector potentials can be used in some cases to simplify the calculation of the electric amd magnetic fields, specially when dealing with antennas and waveguides.
		
		\item \textbf{Scalar electric potential:} Electric potential energy that a unit charge test particle has when it is under the influence of the electric field produced by another charge.
		
		\item \textbf{Scalar potential:} A scalar field whose gradient is a given vector field.
		
		\item \textbf{Vector potential:} A vector field whose curl is a given vector field.
		
		\item \textbf{Whittaker potentials:} Two scalar functions F and G of space and time, in terms of whose second partial derivatives the electric displacement D and the magnetic field intensity H can be expressed~\cite{Whittaker}. Analytical expressions are known for F and G in terms of the trajectory (position as function of time) of the electric charges producing the fields. It can be shown that both F and G obey the homogeneous wave equation in space and time with propagation velocity equal to the speed of light.
	\end{list}
	
	
	\item \textbf{Voltage:} A.k.a. electric potential difference, it is the difference in electric potential (i.e. energy of a unit charge test particle) between two points in space under the influence of the electric field of an/other charge/s. In a static electric field it is defined as the work needed per unit of charge to move a test charge between the two points. Its SI unit is the \emph{volt}.
\end{list}


\addcontentsline{toc}{part}{Bibliography}
\begin{thebibliography}{99}
	\bibitem{Vlaen} van Vlaenderen, Koen J. \emph{A generalisation of classical electrodynamics for the prediction of scalar field effects}. 2003 (physics/0305098v1).
	\bibitem{Dea} Dea, Jack. \emph{Fundamental fields and phase information.} Planetary Association for Clean Energy Newsletter, Vol. 4, Number 3.
	\bibitem{dInverno} d'Inverno. \emph{Introducing Einstein's Relativity.} Oxford.
	\bibitem{Griffiths} Griffiths, David J. \emph{Introduction to electrodynamics}, 4th ed. Pearson.
	\bibitem{Whittaker} Whittaker, E. T. \emph{On an expression of the electromagnetic field due to electrons by means of two scalar potential functions}. 1904.
\end{thebibliography}

\end{document}
